\section*{Teoretická část}
ATLAS je částicový detektor v LHC v CERN. Budeme pozorovat vysokoenergetické srážky protonů a identifikovat jejich vzniklé částice.
Částice, které hledáme, mají velmi krátkou dobu života, takže se zaměříme na produkty jejich rozpadu, pomocí kterých určíme jejich invariantní hmotnost a tedy druh částice.

Boson Z má hmotnost \SI{91}{\GeV\per $c^2$} a budeme detekovat jeho rozpad buď na elektron a pozitron, nebo mion a antimion \cite{skripta}.

Higgsův boson má hmotnost \SI{125}{\GeV\per $c^2$} a budeme detekovat jeho rozpad buď na dva bosony Z (následovaný rozpadem každého z nich), nebo na dva fotony \cite{skripta}.

